\documentclass[10pt,a4paper]{article}

\renewcommand{\baselinestretch}{1.0}
\usepackage{cite}
\usepackage[dvips]{graphicx}
\usepackage{psfrag}
\usepackage{color}
\usepackage[cmex10]{amsmath}
\usepackage{amsfonts}
\usepackage[font=footnotesize, captionskip=10pt]{subfig}
\usepackage{tikz}
\usepackage{flushend}
\usepackage{times}
\usepackage[margin=1.5cm]{geometry}
\usepackage[slovak]{babel}
\usepackage[utf8]{inputenc}
\usepackage[T1]{fontenc}

\usepackage{multirow}
\usepackage{colortbl}

\pagestyle{empty}

\hyphenation{net-works}
\newtheorem{remark}{Remark}

\newcommand\pro{\item[{\bf \color{green} +}]}
\newcommand\con{\item[{\bf \color{red} -}]}

\begin{document}

\section{Activity assistant}
Cieľom projektu je vytvoriť náramok - fitness band, ktorý bude schopný automaticky
určiť typ aktivity, prípade určiť či je daná aktivita vykonávaná správne, resp. podľa plánu.
Predpokladá sa overenie rôznych architektúr embedded sieti.

{\bf Projekt požaduje}

\begin{itemize}
    \item návrh hardvéru - pravdepodobne cortex M0+, acc + gyro, oled display, bluetooth
    \item tvorba android aplikácie
    \item tvorba datasetu - pre vybrané športové aktivity bude treba zozbierať dataset
            (beh, lezenie, box, posilovanie, lukostrelba ...)
    \item tvorba modelu siete a trénovanie, zvažuje sa použitie preprocessingu pomocou DCT alebo FFT
    \item validácia riešenia v praktických podmienkach
\end{itemize}

{\bf Zhrnutie}

\begin{itemize}
    \pro skúsenosti s návrhom podobných zariadení
    \pro hotový framework na embedded siete
    \pro uskutočniteľnosť - nie je to megalomanský projekt
    \con tvorba datasetu
    \con miniaturizácia riešenia
    \con nulový vedecký prínos - len inžiniersky
\end{itemize}

\section{Aeris}
Pokračovanie (reštart) projektu Aeris, nie ako produkt dvoch ľudí ale celého týmu.
Predpokladaný ďalší smer v AI sú aktívne konajúci agenti - nie len pasívne entity
merajúce dáta či robiace klasifikáciu, ale aktívne sa zapájajúci do zmeny prostredia.
Cieľom projektu Aeris Reloaded je posunúť ďalej multiagentové systémy, kde správanie
sa agenta nie je vnútené (napevno naprogramované) ale získané / naučené v prostredí.
Chceme ukázať riešenie problémov vyžadujúce spoluprácu viacerích agentov učených pomocou
reinforcement learning.

{\bf Projekt požaduje}

\begin{itemize}
    \item návrh rozhraní virtuálneho prostredia
    \item návrh simulátora - podpora GPU, automatizácia experimentov ...
    \item návrh reálnych robotov - nepovinné
    \item detekcia a trasovanie robotov - nepovinné
\end{itemize}

{\bf Zhrnutie}

\begin{itemize}
    \pro vedecký prínos v čiastočne pozorovateľných markovových rozhodovacích procesoch
    \pro vedecký prínos v kolaboratívnom RL
    \pro vedecký prínos v rekurentnych sietiach pre RL
    \pro prestížny projekt svetového významu
    \con mimoriadne náročný na spoluprácu týmu - už som raz vyhorel keď mi nik nepomohol s Aeris 1
    \con nie je to projekt na tri nedele
\end{itemize}

\end{document}
